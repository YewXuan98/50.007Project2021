% Options for packages loaded elsewhere
\PassOptionsToPackage{unicode}{hyperref}
\PassOptionsToPackage{hyphens}{url}
%
\documentclass[
]{article}
\usepackage{amsmath,amssymb}
\usepackage{lmodern}
\usepackage{iftex}
\ifPDFTeX
  \usepackage[T1]{fontenc}
  \usepackage[utf8]{inputenc}
  \usepackage{textcomp} % provide euro and other symbols
\else % if luatex or xetex
  \usepackage{unicode-math}
  \defaultfontfeatures{Scale=MatchLowercase}
  \defaultfontfeatures[\rmfamily]{Ligatures=TeX,Scale=1}
\fi
% Use upquote if available, for straight quotes in verbatim environments
\IfFileExists{upquote.sty}{\usepackage{upquote}}{}
\IfFileExists{microtype.sty}{% use microtype if available
  \usepackage[]{microtype}
  \UseMicrotypeSet[protrusion]{basicmath} % disable protrusion for tt fonts
}{}
\makeatletter
\@ifundefined{KOMAClassName}{% if non-KOMA class
  \IfFileExists{parskip.sty}{%
    \usepackage{parskip}
  }{% else
    \setlength{\parindent}{0pt}
    \setlength{\parskip}{6pt plus 2pt minus 1pt}}
}{% if KOMA class
  \KOMAoptions{parskip=half}}
\makeatother
\usepackage{xcolor}
\IfFileExists{xurl.sty}{\usepackage{xurl}}{} % add URL line breaks if available
\IfFileExists{bookmark.sty}{\usepackage{bookmark}}{\usepackage{hyperref}}
\hypersetup{
  hidelinks,
  pdfcreator={LaTeX via pandoc}}
\urlstyle{same} % disable monospaced font for URLs
\usepackage{color}
\usepackage{fancyvrb}
\newcommand{\VerbBar}{|}
\newcommand{\VERB}{\Verb[commandchars=\\\{\}]}
\DefineVerbatimEnvironment{Highlighting}{Verbatim}{commandchars=\\\{\}}
% Add ',fontsize=\small' for more characters per line
\newenvironment{Shaded}{}{}
\newcommand{\AlertTok}[1]{\textcolor[rgb]{1.00,0.00,0.00}{\textbf{#1}}}
\newcommand{\AnnotationTok}[1]{\textcolor[rgb]{0.38,0.63,0.69}{\textbf{\textit{#1}}}}
\newcommand{\AttributeTok}[1]{\textcolor[rgb]{0.49,0.56,0.16}{#1}}
\newcommand{\BaseNTok}[1]{\textcolor[rgb]{0.25,0.63,0.44}{#1}}
\newcommand{\BuiltInTok}[1]{#1}
\newcommand{\CharTok}[1]{\textcolor[rgb]{0.25,0.44,0.63}{#1}}
\newcommand{\CommentTok}[1]{\textcolor[rgb]{0.38,0.63,0.69}{\textit{#1}}}
\newcommand{\CommentVarTok}[1]{\textcolor[rgb]{0.38,0.63,0.69}{\textbf{\textit{#1}}}}
\newcommand{\ConstantTok}[1]{\textcolor[rgb]{0.53,0.00,0.00}{#1}}
\newcommand{\ControlFlowTok}[1]{\textcolor[rgb]{0.00,0.44,0.13}{\textbf{#1}}}
\newcommand{\DataTypeTok}[1]{\textcolor[rgb]{0.56,0.13,0.00}{#1}}
\newcommand{\DecValTok}[1]{\textcolor[rgb]{0.25,0.63,0.44}{#1}}
\newcommand{\DocumentationTok}[1]{\textcolor[rgb]{0.73,0.13,0.13}{\textit{#1}}}
\newcommand{\ErrorTok}[1]{\textcolor[rgb]{1.00,0.00,0.00}{\textbf{#1}}}
\newcommand{\ExtensionTok}[1]{#1}
\newcommand{\FloatTok}[1]{\textcolor[rgb]{0.25,0.63,0.44}{#1}}
\newcommand{\FunctionTok}[1]{\textcolor[rgb]{0.02,0.16,0.49}{#1}}
\newcommand{\ImportTok}[1]{#1}
\newcommand{\InformationTok}[1]{\textcolor[rgb]{0.38,0.63,0.69}{\textbf{\textit{#1}}}}
\newcommand{\KeywordTok}[1]{\textcolor[rgb]{0.00,0.44,0.13}{\textbf{#1}}}
\newcommand{\NormalTok}[1]{#1}
\newcommand{\OperatorTok}[1]{\textcolor[rgb]{0.40,0.40,0.40}{#1}}
\newcommand{\OtherTok}[1]{\textcolor[rgb]{0.00,0.44,0.13}{#1}}
\newcommand{\PreprocessorTok}[1]{\textcolor[rgb]{0.74,0.48,0.00}{#1}}
\newcommand{\RegionMarkerTok}[1]{#1}
\newcommand{\SpecialCharTok}[1]{\textcolor[rgb]{0.25,0.44,0.63}{#1}}
\newcommand{\SpecialStringTok}[1]{\textcolor[rgb]{0.73,0.40,0.53}{#1}}
\newcommand{\StringTok}[1]{\textcolor[rgb]{0.25,0.44,0.63}{#1}}
\newcommand{\VariableTok}[1]{\textcolor[rgb]{0.10,0.09,0.49}{#1}}
\newcommand{\VerbatimStringTok}[1]{\textcolor[rgb]{0.25,0.44,0.63}{#1}}
\newcommand{\WarningTok}[1]{\textcolor[rgb]{0.38,0.63,0.69}{\textbf{\textit{#1}}}}
\setlength{\emergencystretch}{3em} % prevent overfull lines
\providecommand{\tightlist}{%
  \setlength{\itemsep}{0pt}\setlength{\parskip}{0pt}}
\setcounter{secnumdepth}{-\maxdimen} % remove section numbering
\ifLuaTeX
  \usepackage{selnolig}  % disable illegal ligatures
\fi

\title{ML Project report}
\author{Daniel Low \\ 1004372 \and Tan Li Yuan \\1004326 \and Teo Yew Xuan\\1004452}
\date{}

\begin{document}

\maketitle

\section{To Run}

To run script, cd into the root directory of the project \& type the
following command:

\begin{Shaded}
\begin{Highlighting}[]
\ExtensionTok{./script.sh}
\end{Highlighting}
\end{Shaded}

Alternatively (or if using a non-unix system), in the root directory of
the project, run the following:

\begin{Shaded}
\begin{Highlighting}[]
\ExtensionTok{python3}\NormalTok{ hmm\_part\_1.py}
\ExtensionTok{python3}\NormalTok{ hmm\_part\_2.py}
\ExtensionTok{python3}\NormalTok{ hmm\_part\_3.py}
\ExtensionTok{python3}\NormalTok{ hmm\_part\_4.py}
\ExtensionTok{python3}\NormalTok{ hmm\_part\_4b.py}

\CommentTok{\# run on test data}
\ExtensionTok{python3}\NormalTok{ hmm\_part\_4.py }\AttributeTok{{-}t}\NormalTok{ train }\AttributeTok{{-}i}\NormalTok{ test.in}

\ExtensionTok{python3}\NormalTok{ EvalScript/evalResult.py RU/dev.p1.out RU/dev.out }\OperatorTok{\textgreater{}}\NormalTok{ RU/1.out}
\ExtensionTok{python3}\NormalTok{ EvalScript/evalResult.py ES/dev.p1.out ES/dev.out }\OperatorTok{\textgreater{}}\NormalTok{ ES/1.out}

\CommentTok{\# repeat for all the output files}
\end{Highlighting}
\end{Shaded}

\hypertarget{part-2-3}{%
\section{Part 2 \& 3}\label{part-2-3}}

In our transition step, instead of only keeping track of the best
sequence found, we keep track of the all possible sequences occuring
after the previous step, and we truncate each step to the top \texttt{i}
values (\texttt{i\ =\ 1} in part 2 and \texttt{i\ =\ 5} in part 3).

We use log likelihood to avoid floating point precision issues.

To handle the cases where no path can be found with a score of more than
0 (which happens for example when a word we have only observed as
\texttt{I-positive} occurs after a word only observed as \texttt{O},
since $t(\texttt{I-positive} | \texttt{O}) = 0$), we set $\log(0)$
to be a large negative number ($-1000000$) so that in such cases,
we consider the ``least impossible value'' (likelihood scores where we
have to multiply by the least number of zeroes, followed by those with
the same number of zeroes where we would have a higher score if we did
not multiply by those zeroes).

In practice our log score (when there is a feasible solution) is in the
range $[-10^3, 0]$, so this should not affect the cases
where a feasible solution can be found.

\hypertarget{part-4}{%
\section{Part 4}\label{part-4}}

A disadvantage of the HMM is that it only remembers the previous state.
In real life, when we interpret the sentiment of a word we use much more
of the surrounding context to make our judgements.

Hence we extend the HMM by hypothesizing that each new state does not
solely depend on the previous state, but also the state before it. In
this model, instead of having transmission parameters
\(t(y_n | y_{n-1})\) we have a new transmission parameter
\(t(y_n | y_{n-1}, y_{n-2})\).

To find the tag sequence that gives us the maximum score, we extend
Viterbi's algorithm to keep track of the maximum score so far given the
current tag and the previous tag. Letting this be
\(b_{i, y_i, y_{i-1}}\), we have the recurrence

\[
b_{i, y_i, y_{i-1}} = \max_{y_{i-2}} b_{i-1, y_{i-1}, y_{i-2}} e(x_i | y_i) t(y_i | y_{i-1}, y_{i-2})
\]

In addition, we no longer try to handle unknown words in the same way as
before. This is because there are a lot more values tagged as \texttt{O}
than there are for instance \texttt{I-negative}, and because we estimate
the emission parameters for unknown words as
$e(\texttt{\#UNK\#} | Y) = \frac{1}{\text{Count}(Y)}$ the parameter
$e(\texttt{\#UNK\#} | \texttt{O})$ is a lot less than the parameter
$e(\texttt{\#UNK\#} | \texttt{I-negative})$. In practice there is not
much reason for the assignment of these scores, and by handling each of
them as ``impossible'' cases (as explained in part 2/3) we are
effectively assigning them the same emission parameters for each
possible tag. We have found that our F values increase quite
dramatically due to this change, as realistically we should expect that
most unknown values should be tagged as \texttt{O} (rather than as the
other tags, which are much rarer).

Our Sentiment F scores improve on part 2 by about 0.2 in their
evaluation of both the ES and RU development sets.

\begin{verbatim}
# ES

#Entity in gold data: 255
#Entity in prediction: 215

#Correct Entity : 139
Entity  precision: 0.6465
Entity  recall: 0.5451
Entity  F: 0.5915

#Correct Sentiment : 112
Sentiment  precision: 0.5209
Sentiment  recall: 0.4392
Sentiment  F: 0.4766

# RU

#Entity in gold data: 461
#Entity in prediction: 353

#Correct Entity : 239
Entity  precision: 0.6771
Entity  recall: 0.5184
Entity  F: 0.5872

#Correct Sentiment : 163
Sentiment  precision: 0.4618
Sentiment  recall: 0.3536
Sentiment  F: 0.4005
\end{verbatim}

We have attempted to extend this model such that our transitions depend
on the previous 3 states instead (as seen in the files labeled part 4b),
but we have found that this does not meaningfully affect our F scores.

\end{document}
